\documentclass[a4paper,12pt]{article}

% Pakiety
\usepackage[utf8]{inputenc} % Kodowanie UTF-8
\usepackage[T1]{fontenc}   % Kodowanie czcionki
\usepackage{lmodern}       % Poprawione czcionki
\usepackage{geometry}      % Ustawienia marginesów
\geometry{margin=1in}      % Marginesy 1 cal
\usepackage{graphicx}      % Wstawianie obrazków
\usepackage{hyperref}      % Hiperłącza
\usepackage{amsmath}       % Symbole matematyczne
\usepackage{titlesec}      % Formatowanie sekcji

% Metadane
\title{Przykładowa Strona w LaTeX}
\author{Jan Kowalski \and Maria Nowak}
\date{\today}

% Dokument
\begin{document}

% Strona tytułowa
\maketitle
\newpage

% Spis treści
\tableofcontents
\newpage

% Sekcja 1
\section{Wprowadzenie}
LaTeX jest systemem składu tekstu używanym do tworzenia profesjonalnych dokumentów. Ten szablon pokazuje, jak rozpocząć pracę.

% Sekcja 2
\section{Podstawy LaTeXa}
\subsection{Tworzenie sekcji}
Sekcje tworzy się za pomocą poleceń \texttt{\textbackslash section}, \texttt{\textbackslash subsection} i \texttt{\textbackslash subsubsection}. 

\subsection{Matematyka}
LaTeX umożliwia zapis równań matematycznych, np.:
\begin{equation}
E = mc^2
\end{equation}

% Sekcja 3
\section{Obrazki i tabele}
\subsection{Wstawianie obrazków}
Aby wstawić obrazek, użyj:\newline
\begin{verbatim}
\includegraphics[width=\linewidth]{nazwa_pliku.png}
\end{verbatim}

\subsection{Tabele}
Przykład tabeli:
\begin{table}[h!]
    \centering
    \begin{tabular}{|c|c|c|}
        \hline
        Nagłówek 1 & Nagłówek 2 & Nagłówek 3 \\
        \hline
        A          & B          & C          \\
        \hline
    \end{tabular}
    \caption{Przykładowa tabela}
    \label{tab:example}
\end{table}

% Sekcja 4
\section{Podsumowanie}
Ten dokument prezentuje podstawowe możliwości LaTeXa. Więcej informacji można znaleźć w oficjalnej dokumentacji.

\end{document}
